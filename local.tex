

% Coordenador(a) atual do Curso
\coursecoord{Prof. Dr. Marcelo Götz}
\coursecoordgender
\coursecoordtitle{Doutor por \textit{Universität Paderborn} -- Paderborn, Alemanha }

% o nome do curso pode ser redefinido (ex. para Monografias)
%
%% \courseacronym{CCA}
\course{Engenharia de Controle e Automação}
\graduationtitle{Bacharel em Engenharia de Controle e Automação}
\tcc{Trabalho de Conclusão de Curso -- CCA}

%\universitydivision{Escola de Engenharia}
%\university{Universidade Federal...}

%% for a report
%
%\courseundef
%\department{DELAE - Dept. de Sistemas Elétricos e Energia}
%\class{ENG100xy}{some thing else}
%\subject{Segundo trabalho da disciplina}

% o local de realização do trabalho pode ser especificado (ex. para Monografias)
% com o comando \location:
%\location{São José dos Campos}{SP}


%%%%%%%%%%%%%%%%%%%%%%%%%%%%%%%%%%%%%%
%%%%%%%%%%%%%%%%%%%%%%%%%%%%%%%%%%%%%%
%%%%%%%%%%%%%%%%%%%%%%%%%%%%%%%%%%%%%%


% Informações gerais
%
%\title{Adaptação do Método VRFT para Rejeição de Modos Ressonantes no Ajuste de Controlador em Fonte de Alimentação Ininterrupta}
%\title{Aplicação do Método VRFT para Sintonia de Controlador Proporcional-Múltiplo-Ressonante em Fonte de Alimentação Ininterrupta}
%\title{Projeto de Controlador Proporcional-Múltiplo-Ressonante Ajustado pelo Método VRFT para Rejeição de Modos Ressonantes em Fonte de Alimentação Ininterrupta}
%\title{Projeto de Controlador Proporcional-Múltiplo-Ressonante Ajustado pelo Método VRFT para Rejeição de Harmônicas em Fonte de Alimentação Ininterrupta}
%\title{Projeto de Controlador Proporcional-Múltiplo-Ressonante Sintonizado pelo Método VRFT para Rejeição de Harmônicas em Fonte de Alimentação Ininterrupta}
\title{Projeto de Controlador Proporcional-Múltiplo-Ressonante Sintonizado por VRFT para Rejeição de Harmônicas em Fonte de Alimentação Ininterrupta}

\author{de Paoli Beal}{Guilherme}
\authorinfo{00243703}{guibeal96@gmail.com}
% alguns documentos podem ter varios autores:
%\author{Flaumann}{Frida Gutenberg}
%\author{Flaumann}{Klaus Gutenberg}

% orientador
\advisor[Profa. Dra.]{Campestrini}{Lucíola}
\advisorinfo{UFRGS}{Doutora pela Universidade Federal do Rio Grande do Sul -- Porto Alegre, Brasil}{luciola@ufrgs.br}{4472}
\advisorgender[f] %% the default is [m], other option is [f]

% O comando \advisorwidth pode ser usado para ajustar o tamanho do campo
% destinado ao nome do orientador, de forma a evitar que ocupe mais de uma linha 
%\advisorwidth{0.60\textwidth}

% obviamente, o co-orientador é opcional
\coadvisor[Dr.]{Lorenzini}{Charles}
%\coadvisorinfo{UFRGS}{Doutor pela Universidade Federal do Rio Grande do Sul -- Porto Alegre, Brasil}{email}{ramal}
\coadvisorinfo{UFRGS}{Doutor pela Universidade Federal do Rio Grande do Sul -- Porto Alegre, Brasil}{}{}
\coadvisorgender[m] %% the default is [m], other option is [f]

% banca examinadora
\examiner[Prof.~Dr.]{Tergolina Salton}{Aurélio}
\examinerinfo{UFRGS}{Doutor por \textit{The University of Newcastle} -- Newcastle, Austrália}{aurelio.salton@ufrgs.br}{4281}

\examiner[Prof.~Dr.]{Vieira Flores}{Jeferson}
\examinerinfo{UFRGS}{Doutor pela Universidade Federal do Rio Grande do Sul -- Porto Alegre, Brasil}{jeferson.flores@ufrgs.br}{3291}

% suplentes da banca examinadora (apenas para alguns formulários)
\altexaminer[Prof.~Dr.]{Eckhard}{Diego}
\altexaminerinfo{UFRGS}{Doutor pela Universidade Federal do Rio Grande do Sul -- Porto Alegre, Brasil}{diegoeck@ufrgs.br}{6219}


%% resumo do trabalho (para o formulário de renovação de requerimento de matrícula.
%%
\tccbrief{algo a ser feito...}
\tccadvisorsreview{Parecer final do Orientador.}
\tcccoadvisorbrief{justificativa para ter-se um co-orientador...}


% palavras-chave
% iniciar todas com letras minúsculas, exceto no caso de abreviaturas
%
\keyword{Controlador Proporcional-Múltiplo-Ressonante}
\keyword{Controle Baseado em Dados}
\keyword{Controle em Cascata}
\keyword{Fonte de Alimentação Ininterrupta}
\keyword{Rejeição de Harmônicas}
\keyword{\textit{Virtual Reference Feedback Tuning}}

% a data deve ser a da defesa; se nao especificada, são gerados
% mes e ano correntes
\date{maio}{2021}


